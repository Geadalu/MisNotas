\ifspanish
	\selectlanguage{spanish}
\else
	\selectlanguage{english}
\fi

% -------------------------
%
% AGRADECIMIENTOS (recomendable máx. 1 pág.)
%
% -------------------------
\cleardoublepage
\phantomsection % OJO: Necesario con hyperref
\pdfbookmark[0]{Agradecimientos}{idx_agrad}% idx_agrad.0 % Bookmark en PDF

\chapter*{Agradecimientos} % Opción con * para que no aparezca en TOC ni numerada

Principalmente, me gustaría agradecer la realización de este trabajo a mi madre, quien me levantó, apoyó y alentó innumerables veces desde que comencé la carrera, a pesar de todos mis tropiezos y caídas. Gracias también, mamá, por inspirarme para elegir este trabajo.

De manera más general, me gustaría agradecer a mi familia, que igualmente me apoyó y me escuchó cuando más lo necesitaba, especialmente a mi hermano José por aguantar vivir conmigo en Ciudad Real. Quiero pensar que nos hemos apoyado mutuamente todo este tiempo.

A mi tutora Carmen por haberme guiado a lo largo de este proyecto.

Agradecer, por supuesto a Pablo, mi mejor amigo ya para toda la vida, el haberme acompañado durante este largo y sinuoso viaje que ha sido para mí la Universidad, desde el momento en el que nos sentamos juntos en la primera charla que nos dieron hasta las noches locas por Ciudad Real que, junto con Ruth (y en ocasiones Natalia), son completamente inolvidables. Te quiero mucho, y siempre podrás contar conmigo para pasarme memes y gifs de gatitos.

A Juanjo, la absoluta e inmarcesible luz que ilumina mi camino, por apoyarme y animarme en cada uno de mis peores días y disfrutar de los mejores. Por aceptarme tal y como soy y, lo que es más difícil, ser inmensamente feliz con ello. No te doy las gracias por este trabajo sino por EL trabajo que te doy día a día y que, con gusto y sin ningún problema, has parecido aceptar para toda la vida. Y yo no podría ser más feliz.

A toda esta gente por haberme ayudado a caminar, haber pasado momentos junto a mí y haberme hecho reír: Javier Córdoba y su impresionante Wrecking Ball, la maravillosa gente del JuegoJueves entre los que se incluyen Carfer y Álvaro, Alfonsa por haber sido mi vía de escape en la residencia, Javier Sánchez y las horas que le hemos echado al Minecraft, y a Hazurov por hacerme sentir que soy mucho más que una persona aleatoria al otro lado de la pantalla.

Para finalizar, me gustaría agradecer gran parte de este trabajo a StackOverflow y su gran labor didáctica, sin la cual, este trabajo me hubiera llevado el triple de tiempo.



\makeatletter		
\begin{flushright}
	\textit{\@autor}
\end{flushright}
\makeatother