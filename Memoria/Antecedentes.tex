\chapter{Aplicaciones existentes}
\label{cap:aplicacionesexistentes}

En este apartado se describen aplicaciones existentes que permiten al profesorado la gestión del proceso de calificación de su alumnado junto con un análisis comparativo de los mismos.

\section{EducamosCLM}

Esta aplicación web de la Junta de Comunidades de Castilla-La Mancha\cite{educamosclm} es la aplicación de gestión de alumnado por excelencia en los centros educativos públicos. 

Busca dotar de herramientas de gestión y comunicación para el profesorado, el alumnado y los padres mediante un entorno seguro, flexible e intuitivo. Además también posee un módulo para realizar trámites en los centros.

\begin{figure}[h]
\centering\includegraphics[width=1\linewidth]{figs/educamosCLM2.png}
\caption{Crear un examen en EducamosCLM.\cite{educamosclmyoutube}}
\label{Fig:educamosCLM}
\end{figure}

A continuación se muestran las características principales de EducamosCLM comparadas con las de esta aplicación.

\textbf {Seguimiento del curso.}
    En este apartado, los docentes pueden publicar las notas de sus alumnos para que estos y los padres las vean en cualquier momento, así como las faltas de asistencia y la trayectoria escolar que lleva el alumno durante el curso.
    En la vista de los alumnos, estos podrán subir sus trabajos on-line, que le aparecerán al docente para que pueda descargarlos e introducir la calificación en el sistema. Este sistema también permite a los alumnos pedir tutorías con los docentes.


\section{Google Classroom}

Google Classroom es una aplicación para navegador web y para smartphone\cite{googleclassroom} desarrollada por Google que permite la comunicación entre docentes y alumnos, así como la gestión y organización de trabajos mediante Google Drive.

\begin{figure}[h]
\centering\includegraphics[width=1\linewidth]{figs/googleclassroom.png}
\caption{Crear una tarea en Google Classroom.\cite{googleclassroomyoutube}}
\label{Fig:googleclassroom}
\end{figure}

A continuación se muestran las características principales de Google Classroom comparadas con las de esta aplicación.

\textbf {Orientado a docentes y alumnos.} Principalmente, Google Classroom está pensado tanto para docentes como para alumnos, por lo que es muy completo. Tiene herramientas para programar entregas, reuniones y para que el docente pueda comunicarse mediante mensajes de texto con un alumno o con toda la clase.

\textbf {Archivos en la nube.} Todos los archivos que se suben van directamente a Google Drive. De esta forma se almacenan todos en el mismo sitio, pero de forma ordenada, y cada alumno y docente puede acceder a estos archivos mediante una cuenta Google aceptada en el curso. 

\textbf {Personalizable mediante aplicaciones externas.} Google Classroom es una aplicación flexible: permite la integración de aplicaciones como Classcraft, Pear Deck o Quizizz, posibilitando una completa personalización de la experiencia tanto para los alumnos como para los docentes. Debido a la generalización de la herramienta, es muy versátil y se puede usar para cualquier curso, tanto de primaria como de secundaria.
	

\section{Additio}
\label{sec:additio}

Additio es una aplicación para navegador web y smartphone\cite{additio} que permite gestionar las notas del alumnado y las competencias que tiene cada metodología, planificar las clases y la comunicación con padres y alumnos.

\begin{figure}[h]
\centering\includegraphics[width=1\linewidth]{figs/additio.png}
\caption{Gestión de notas en Additio.\cite{additioyoutube}}
\label{Fig:additio}
\end{figure}

A continuación se muestran las características principales de Additio comparadas con las de la aplicación desarrollada en este documento.

Additio está diseñada tanto para docentes como para Centros y permite, como las anteriores, gestionar notas y trabajos, la asistencia a clase y la comunicación entre padres, docentes y alumnos. Sin embargo, tiene una característica nueva que no se había encontrado en las aplicaciones descritas anteriormente: la posible introducción de competencias para las pruebas.

\textbf {Aplicación on-line multiplataforma.} Additio puede usarse en smartphone, tablet u ordenador, lo que permite un mayor seguimiento de las actividades, independiente de la localización del docente.
 
\textbf {Calendario y agenda.} Additio también contiene un calendario y una agenda para establecer citas con alumnos y padres.

\textbf {Pocas opciones de personalización de interfaz.} Additio no permite modificar la interfaz o los colores a gusto del usuario.


\section{Análisis y conclusiones}

Para concluir este capítulo, se realiza un pequeño análisis de la diferencia entre las aplicaciones elegidas.

A continuación se listan más diferencias:
\begin{itemize}
\item \textbf{Personalización de la interfaz.} Ninguna de las aplicaciones investigadas posee un gran grado de personalización. Aunque a primera vista podría parecer de poca importancia, la realidad es que si la interfaz presenta dificultades para ser leída, la usabilidad se deteriora y el usuario se siente menos cercano a la hora de usar la aplicación.
\item \textbf{Sin conexión a Internet.} Todas las aplicaciones investigadas tienen conexión a Internet, y aunque el hecho de no tenerla a priori pueda parecer un paso atrás en el desarrollo de una aplicación, la realidad es otra. La decisión de desarrollar una aplicación de escritorio sin conexión a Internet tiene varias ventajas, entre ellas el acceso en cualquier lugar, aunque no se posea Internet; una mayor seguridad de los datos debido a la imposibilidad de fugas de datos personales y el centrado del foco de atención del docente en su trabajo, ya que no tendrá que salir de la aplicación para realizar ningún otro trámite.
\item \textbf{Aplicación de escritorio.} Unida con las razones anteriores, las aplicaciones de escritorio son más personalizables, y si bien pueden requerir de actualizaciones manuales, esto permite que se puedan realizar a la vez para todos los usuarios del mismo Centro de tal forma que todos los docentes tengan siempre la misma versión de la aplicación.
\end{itemize}

Como conclusión final, se considera que las aplicaciones investigadas poseen una curva de aprendizaje demasiado elevada y un nivel pobre de personalización a gusto del usuario.