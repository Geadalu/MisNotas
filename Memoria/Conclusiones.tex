\chapter{Conclusiones}
\label{cap:conclusiones}
En este último capítulo se habla de las conclusiones extraídas de este trabajo y se proponen algunas vías para continuarlo o mejorarlo.

\section{Conclusiones}
En este trabajo se ha desarrollado una herramienta intuitiva y usable para una cómoda calificación de los docentes a su alumnado. Se han conseguido todos los objetivos propuestos en el capítulo \ref{cap:objetivo} e incluso se han añadido algunas funcionalidades nuevas que no se pensaron al principio.

En comparación a las aplicaciones existentes de las que se habló en el capítulo \ref{aplicacionesexistentes}, se considera que el resultado de este proyecto las ha mejorado en los siguientes puntos:
\begin{enumerate}
	\item \textbf{Privacidad de los datos}. Todas las aplicaciones que se investigaron tenían acceso a Internet. Si bien esto, para sus especificaciones, era necesario (debido a que todas ellas implementaban la capacidad de comunicación entre docente y alumnado), también supone un mínimo riesgo de filtración de datos. La aplicación desarrollada, al ser de escritorio, no encuentra ese problema y sus datos están completamente seguros.
	\item \textbf{Alta personalización de la interfaz gráfica}. A la hora de trabajar con un ordenador, hay algunas personas que necesitan aumentar el tamaño de la letra, o crear un mayor contraste entre los elementos que se muestran en la pantalla. Esta aplicación lo permite, posibilitando una sesión de trabajo lo más cómoda y agradable posible.
	\item \textbf{Control de competencias}. Si bien Additio \ref{sec:additio} era la única aplicación que permitía un control de competencias, en la aplicación desarrollada se permiten visualizar con mayor claridad en cualquier momento mediante la funcionalidad "Informe del trimestre".
\end{enumerate}

Para terminar, comentar que gracias a la prueba de usabilidad realizada, se puede concluir que el desarrollo ha sido un éxito, aunque podría mejorar en varios aspectos que se discuten en la sección a continuación.
	

\section{Propuestas de trabajo futuro}
\label{sec:trabajofuturo}
Como posible continuación del desarrollo de esta aplicación, se proponen a continuación varios puntos que añaden funcionalidades nuevas o mejoran las existentes.

Estas son ideas que han ido surgiendo a lo largo del desarrollo, y que podrían beneficiar a la aplicación de distintas maneras.

\begin{itemize}
	\item \textbf{Mejora de la interfaz gráfica}. En general, el aspecto de la aplicación es primitivo y simple. Esto es un importante aspecto a tener en cuenta puesto que podría aumentar el número de docentes que desearan utilizar esta aplicación, además de mejorar la experiencia de usuario.
	\item \textbf{Adición del rol de administrador}. En el momento de finalización de este proyecto solo hay un tipo de usuario que puede acceder a la aplicación, y este es el maestro. Sin embargo, se ha considerado que el sistema podría beneficiarse de la existencia de un rol de administrador, que accediera a una pantalla especial para cargar nuevas asignaturas y nuevos cursos al docente, y para que inicializara las aplicaciones para los docentes nuevos en el centro o cada inicio de curso.
	\item \textbf{Más opciones de personalización}. Varios de los comentarios recogidos en la prueba de usabilidad [\ref{sub:pruebausabilidad}] se orientan a la personalización. Se podría implementar un sistema para calificar de manera diferente, en lugar de solo con un número para la nota y un comentario, de manera menos sistemática para tener en cuenta otros aspectos aparte de la nota.
	\item \textbf{Posibilidad de apuntar si se ha hecho la tarea}. También se podría implementar un sistema que permita al docente apuntar los días que el alumno ha hecho la tarea para casa e incluso calificarla. También, en la propia ventana de calificación de tareas, si el alumno ha terminado o no haciendo esa tarea.
\end{itemize}
