\chapter{Conclusiones}
\label{cap:conclusiones}
En este último capítulo se habla de las conclusiones extraídas de este trabajo y se proponen algunas vías para continuarlo o mejorarlo.

\section{Conclusiones}
En este trabajo se ha desarrollado una herramienta intuitiva y usable para la calificación de los docentes a su alumnado. Se han conseguido todos los objetivos propuestos en el capítulo \ref{cap:objetivo} e incluso se han añadido algunas funcionalidades nuevas que no se pensaron al principio: las estadísticas y las gráficas que las acompañan y el informe general del trimestre.

En comparación a las aplicaciones existentes de las que se habló en el capítulo \ref{cap:aplicacionesexistentes}, se considera que el resultado de este proyecto las ha mejorado en los siguientes puntos:
\begin{enumerate}
	\item \textbf{Privacidad de los datos}. Todas las aplicaciones que se investigaron tenían acceso a Internet. Si bien esto, para sus especificaciones, era necesario (debido a que todas ellas implementaban la capacidad de comunicación entre docente y alumnado), también supone un mínimo riesgo de filtración de datos. La aplicación desarrollada, al ser de escritorio, no encuentra ese problema y sus datos están completamente seguros.
	\item \textbf{Alta personalización de la interfaz gráfica}. A la hora de trabajar con un ordenador, hay algunas personas que necesitan aumentar el tamaño de la letra, o crear un mayor contraste entre los elementos que se muestran en la pantalla. Esta aplicación lo permite, posibilitando una sesión de trabajo lo más cómoda y agradable posible.
	\item \textbf{Control de competencias}. Si bien Additio \ref{sec:additio} era la única aplicación que permitía un control de competencias, en la aplicación desarrollada se permiten visualizar con mayor claridad en cualquier momento mediante la funcionalidad ''Informe del trimestre''.
\end{enumerate}

En la tabla \ref{tab:Tabla1} se muestra un resumen de esta aplicación con las aplicaciones estudiadas en el capítulo \ref{cap:aplicacionesexistentes}. Las celdas con un asterisco significan que esa funcionalidad podría llegar a existir adaptando otras funcionalidades de la aplicación, o que solo está disponible para algunos Centros.

En general, se puede ver que el punto en el que más se diferencia la aplicación desarrollada con las que se han investigado es la dificultad de uso. Si bien es cierto que las aplicaciones mencionadas poseen más funcionalidades y son más flexibles, debido a esto tienen un grado de dificultad mucho mayor que la aplicación desarrollada.


\begin{table}[H]
\caption{Comparaciones entre las aplicaciones}
\label{tab:Tabla1}
\begin{tabular}{c|c|c|c|
>{\columncolor[HTML]{FFE1BD}}c|}
\cline{2-5}
                                                                                                        & \textbf{EducamosCLM} & \textbf{\begin{tabular}[c]{@{}c@{}}Google\\ Classroom\end{tabular}} & \textbf{Additio} & \textbf{\begin{tabular}[c]{@{}c@{}}Aplicación\\ desarrollada\end{tabular}} \\ \hline
\multicolumn{1}{|c|}{\textbf{Tipo de aplicación}}                                                       & Web                  & Web y smartphone                                                    & Web y smartphone & Escritorio                                                                 \\ \hline
\multicolumn{1}{|c|}{\textbf{\begin{tabular}[c]{@{}c@{}}Gestión de las\\ calificaciones\end{tabular}}}  & Sí                   & Sí                                                                  & Sí               & Sí                                                                         \\ \hline
\multicolumn{1}{|c|}{\textbf{\begin{tabular}[c]{@{}c@{}}Gestión de\\ tareas\end{tabular}}}              & Sí                   & Sí                                                                  & Sí               & Sí                                                                         \\ \hline
\multicolumn{1}{|c|}{\textbf{\begin{tabular}[c]{@{}c@{}}Gestión del\\ alumnado\end{tabular}}}           & No                   & Sí                                                                  & Sí               & Sí                                                                         \\ \hline
\multicolumn{1}{|c|}{\textbf{\begin{tabular}[c]{@{}c@{}}Gestión de las\\ competencias\end{tabular}}}    & *                    & No                                                                  & Sí               & Sí                                                                         \\ \hline
\multicolumn{1}{|c|}{\textbf{\begin{tabular}[c]{@{}c@{}}Sacar informes\\ de las notas\end{tabular}}}    & No                   & Sí                                                                  & Sí               & Sí                                                                         \\ \hline
\multicolumn{1}{|c|}{\textbf{\begin{tabular}[c]{@{}c@{}}Personalización\\ de la interfaz\end{tabular}}} & No                   & No                                                                  & No               & Sí                                                                         \\ \hline
\end{tabular}
\end{table}



Para terminar, gracias a la prueba de usabilidad realizada se puede concluir que el desarrollo ha sido un éxito, aunque podría mejorar en varios aspectos que se discuten en la sección \ref{sec:trabajofuturo}.
	
\section{Justificación de las competencias}
Este trabajo cumple con las siguientes competencias de la rama de Computación:

\subsection{CM3}
Esta competencia se define como \textit{la capacidad para evaluar la complejidad computacional de un problema, conocer estrategias algorítmicas que puedan conducir a su resolución y recomendar, desarrollar e implementar aquella que garantice el mejor rendimiento de acuerdo con los requisitos establecidos.}

Para cumplir con esta competencia se ha investigado y diseñado el código de forma que se ejecuten las instrucciones mínimas posibles en sus algoritmos, cuando ha sido posible. Se pueden ver ejemplos de la aplicación de esta competencia en la sección \ref{sub:implementacion}.


\subsection{CM6}
Esta segunda competencia describe \textit{la capacidad para desarrollar y evaluar sistemas interactivos y de presentación de información compleja y su aplicación a la resolución de problemas de diseño de interacción persona computadora.}

El cumplimiento de esta competencia ha sido alcanzado mediante las diversas iteraciones que han ido teniendo los diseños a lo largo de la etapa de elaboración del \gls{pud}. Estas iteraciones han permitido refinar los diseños para adecuarse a las peticiones del usuario al implementarlos más adelante.

Además, dado que el proyecto ha consistido en diseñar interfaces que resuelvan el problema expuesto en su objetivo, dotarlas de información mediante una base de datos y posteriormente, evaluarlas mediante una prueba de usabilidad con 34 usuarios potenciales, también se considera que se ha cumplido esta competencia.


\section{Propuestas de trabajo futuro}
\label{sec:trabajofuturo}
Como posible continuación del desarrollo de esta aplicación, se proponen a continuación varios puntos que añaden funcionalidades nuevas o mejoran las existentes, parte de ellas extraídas de las aportaciones de los evaluadores.

\begin{itemize}
	\item \textbf{Mejora de la interfaz gráfica}. En general, el aspecto de la aplicación es primitivo y simple. Esto es un importante aspecto a tener en cuenta puesto que podría aumentar el número de docentes que desearan utilizar esta aplicación, además de mejorar la experiencia de usuario.
	\item \textbf{Adición del rol de administrador}. En el momento de finalización de este proyecto solo hay un tipo de usuario que puede acceder a la aplicación, y este es el maestro. Sin embargo, se ha considerado que el sistema podría beneficiarse de la existencia de un rol de administrador, que accediera a una pantalla especial para cargar nuevas asignaturas y nuevos cursos al docente, y para que inicializara las aplicaciones para los docentes nuevos en el centro o cada inicio de curso.
	\item \textbf{Más opciones de personalización}. Varios de los comentarios recogidos en la prueba de usabilidad de la sección \ref{sub:pruebausabilidad} se orientan a la personalización. Se podría implementar un sistema para calificar de manera diferente, en lugar de solo con un número para la nota y un comentario, de manera menos sistemática para tener en cuenta otros aspectos aparte de la nota.
	\item \textbf{Posibilidad de apuntar si se ha hecho la tarea}. También se podría implementar un sistema que permita al docente apuntar los días que el alumno ha hecho la tarea para casa e incluso calificarla. También, en la propia ventana de calificación de tareas, si el alumno ha terminado o no haciendo esa tarea.
\end{itemize}
