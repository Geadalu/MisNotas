\chapter{Introducción}
\label{cap:Introduccion}

Este trabajo va dirigido a los maestros de \gls{eso}, y consiste en la creación de una herramienta software donde puedan almacenar los datos de sus alumnos así como sus calificaciones a lo largo del curso.

A día de hoy, la mayoría de los maestros siguen usando hojas de cálculo para almacenar las calificaciones de sus alumnos. Esto, aunque es una mejor opción que hacerlo a mano, todavía dista mucho de ser una herramienta fácil de usar para cualquier persona, transparente para el maestro y visualmente agradable.

Para ello, se quiere crear una aplicación que, si bien permitirá a los profesionales docentes hacer el seguimiento de las calificaciones de sus alumnos, lo hará de forma simple e intuitiva, reduciendo el tiempo requerido para la tarea y haciéndola más amena.

\section{Usuarios}

Los usuarios de la aplicación serán los maestros de Educación Secundaria Obligatoria.

\section{Aplicaciones existentes}

En este apartado se describen aplicaciones existentes y actuales que hacen lo mismo que la mía pero la mía es mejor por las razones que voy a exponer a continuación.

\subsection{EducamosCLM}

Esta aplicación web de la Junta de Comunidades de Castilla-La Mancha es la aplicación de gestión de alumnado por excelencia en los colegios públicos. 

Características:

\textbf {Seguimiento del curso.}
    En este apartado, los profesores pueden publicar las notas de sus alumnos para que estos y sus padres las vean en cualquier momento, así como las faltas de asistencia y la trayectoria escolar que lleva el alumno durante el curso.
    En la vista de los alumnos, estos podrán subir sus trabajos online, que le aparecerán al profesor para que pueda descargarlos, calificarlos e introducir dicha calificación en el sistema, así como pedir tutorías con los profesores.
        
En qué podría mejorar mi aplicación respecto a esta:

\begin{enumerate}
    \item \textbf{No depende del Centro}. TeachHelper es una aplicación libre, que no está unida a la Junta de Comunidades. Es una aplicación individual con varios niveles de personalización.
    \item \textbf{Más simple}. Mientras que EducamosCLM tiene características especiales que permiten la comunicación con los padres y un alto nivel de personalización de tareas, TeachHelper es mucho más sencillo, intuitivo y fácil de usar, a la vez que mantiene una alta personalización.
    \item \textbf{Personalizado para cada maestro/a}. TeachHelper tiene una característica única: guarda y aprende de las preferencias y los cursos más usados por cada maestro/a, para sugerirlas la próxima vez que entre a la aplicación.
\end{enumerate}

\subsection{Google Classroom}

Aplicación de navegador y de smartphone desarrollada por Google que permite la comunicación entre profesores y alumnos, así como la gestión y organización de trabajos mediante Google Drive.

Características:

\begin{enumerate}
	\item \textbf{Fácil e intuitivo}. Pensado tanto para profesores como para alumnos, Google Classroom es fácil de usar. Tiene herramientas para programar entregas, reuniones y hablar con todo el grupo de alumnos a la vez mediante texto. Además, permite subir archivos a la nube para ser calificados por los profesores.
	\item \textbf{Todo en la nube}. Todos los archivos que se suben van directamente a Google Drive. De esta forma, se tienen todos juntos, pero ordenados.
	\item \textbf{Integración con otras aplicaciones}. Google Classroom permite la integración de aplicaciones como Classcraft, Pear Deck o Quizizz, permitiendo una completa personalización de la experiencia tanto para los alumnos como para los maestros. Esto le da flexibilidad a la aplicación.
	\item \textbf{Accesible para primaria, secundaria o educación superior}. Debido a la generalización de la herramienta, es muy versátil y se puede usar para cualquier curso.
\end{enumerate}

En qué podría mejorar mi aplicación respecto a esta:

\begin{enumerate}
	\item \textbf{Más orientada a los profesores}. Google Classroom es una herramienta orientada a ayudar a dar clase a los profesores. Tiene varias funcionalidades que permiten, por ejemplo, que varios alumnos trabajen en el mismo documento. Esto sale un poco del propósito de TeachHelper, que es el de organización de los documentos del profesorado, no de las propias clases.
	\item \textbf{Cuenta de Google}. Para usar Google Classroom, tanto el alumno como el profesor necesitan tener una cuenta de Google, y eso es algo que no recomiendan todos los Centros, sobre todo si son públicos, por motivos de seguridad de los datos. Cada usuario de Google Classroom debería hacerse una cuenta específica para usarlo. TeachHelper solo va dedicada a los maestros, y solo ellos deberán registrarse.
	\item \textbf{Personalización}. De nuevo, TeachHelper cuenta con la característica de la personalización dependiendo del maestro/a que Classroom no tiene, por el enfoque de la aplicación.
\end{enumerate}
	

\subsection{Additio}

Aplicación de navegador y de smartphone que permite gestionar las notas del alumnado y las competencias que tiene cada metodología, planificar las clases y la comunicación con padres y alumnos.

Características:

\begin{enumerate}
	\item \textbf{Una aplicación potente}. Additio es, de las aplicaciones de las que hemos hablado hasta ahora, la que más se parece a la idea de TeachHelper. Diseñada para profesores como para Centros, permite gestionar notas y trabajos, la asistencia a clase y la comunicación entre padres, profesores y alumnos. 
	\item \textbf{Apartado de cálculo de competencias}. Una de las características de esta aplicación es que permite establecer las competencias de una prueba, a las que se les puede dar peso para calcular notas.
	\item \textbf{Calendario y agenda}. Additio tiene un calendario y una agenda para establecer citas con alumnos o padres.
	\item \textbf{Para Centros y para profesores}. Se pueden contratar dos tipos de aplicaciones: una para el Centro y otra para el profesor.
	\item \textbf{Informes}. Permite sacar informes de cualquier alumno para ver la trayectoria a lo largo de los trimestres.
\end{enumerate}

En qué podría mejorar mi aplicación respecto a esta:

\begin{enumerate}
	\item \textbf{Demasiadas características}. Aunque es una aplicación, como hemos mencionado anteriormente, muy potente, quizá ese sea su punto débil: es muy extensa. TeachHelper está enfocada a ayudar con la gestión de las notas del alumnado y es mucho más simple de utilizar.
	\item \textbf{Pensada principalmente para tablet y smartphone}. En su página web mencionan que esta aplicación está pensada para los profesores que usen su tablet o smartphone como forma principal de gestionar sus clases. TeachHelper es una aplicación de escritorio porque a todos los maestros se les da un ordenador portátil en el trabajo.

\end{enumerate}




En la tabla \ref{tab:Tabla1} se muestran las comparaciones de esta aplicación con las aplicaciones estudiadas. Las celdas con un asterisco significan que esa funcionalidad podría llegar a existir adaptando otras funcionalidades de la aplicación, o solo está disponible para algunos Centros.

\begin{table}[h] 
\caption{Comparaciones entre las aplicaciones}
\label{tab:Tabla1}
\begin{tabular}{@{}lcccc@{}}
\toprule
                                                                                                      & \multicolumn{1}{l}{\textbf{EducamosCLM}} & \multicolumn{1}{l}{\textbf{Google Classroom}} & \multicolumn{1}{l}{\textbf{Additio}} & \multicolumn{1}{l}{\textbf{TeachHelper}} \\ \midrule
\multicolumn{1}{|l|}{\textbf{Inicio de sesión}}                                                       & \multicolumn{1}{c|}{x}                   & \multicolumn{1}{c|}{x}                        & \multicolumn{1}{c|}{x}               & \multicolumn{1}{c|}{x}                   \\ \midrule
\multicolumn{1}{|l|}{\textbf{\begin{tabular}[c]{@{}l@{}}Gestión de \\ calificaciones\end{tabular}}}   & \multicolumn{1}{c|}{x}                   & \multicolumn{1}{c|}{x}                        & \multicolumn{1}{c|}{x}               & \multicolumn{1}{c|}{x}                   \\ \midrule
\multicolumn{1}{|l|}{\textbf{\begin{tabular}[c]{@{}l@{}}Gestión del \\ alumnado\end{tabular}}}        & \multicolumn{1}{c|}{}                    & \multicolumn{1}{c|}{x}                        & \multicolumn{1}{c|}{x}               & \multicolumn{1}{c|}{x}                   \\ \midrule
\multicolumn{1}{|l|}{\textbf{\begin{tabular}[c]{@{}l@{}}Gestión de \\ las competencias\end{tabular}}} & \multicolumn{1}{c|}{*}                    & \multicolumn{1}{c|}{}                         & \multicolumn{1}{c|}{x}               & \multicolumn{1}{c|}{x}                   \\ \midrule
\multicolumn{1}{|l|}{\textbf{\begin{tabular}[c]{@{}l@{}}Comentarios \\ sobre alumnado\end{tabular}}}  & \multicolumn{1}{c|}{}                   & \multicolumn{1}{c|}{*}                        & \multicolumn{1}{c|}{*}               & \multicolumn{1}{c|}{x}                   \\ \midrule
\multicolumn{1}{|l|}{\textbf{\begin{tabular}[c]{@{}l@{}}Recomendaciones \\ dinámicas\end{tabular}}}   & \multicolumn{1}{c|}{}                    & \multicolumn{1}{c|}{}                         & \multicolumn{1}{c|}{}                & \multicolumn{1}{c|}{x}                   \\ \midrule
\multicolumn{1}{|l|}{\textbf{\begin{tabular}[c]{@{}l@{}}Informes del \\ alumnado\end{tabular}}}       & \multicolumn{1}{c|}{}                    & \multicolumn{1}{c|}{x}                        & \multicolumn{1}{c|}{x}               & \multicolumn{1}{c|}{x}                   \\ \bottomrule
\end{tabular}
\end{table}