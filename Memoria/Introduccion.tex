
\chapter{Introducción}
\label{cap:Introduccion}
En esta última década hemos vivido una digitalización en todos los ámbitos que nos podamos imaginar: sociales, culturales, médicos, incluso políticos. Existen numerosas aplicaciones informáticas que nos permiten gestionar cómodamente la mayoría de aspectos de nuestras vidas, tanto de manera cotidiana mediante aplicaciones para pagar, pedir transporte, comprar a domicilio, etc, desde nuestro smartphone, como en su aspecto profesional, transformando la mayoría de profesiones hacia la famosa \textit{transformación digital} que estamos empezando a vivir. De hecho, se ha podido notar en el año 2020, una clara tendencia a esta \textit{transformación digital} debido a la pandemia mundial que hemos sufrido. Numerosos negocios de cualquier sector profesional han tenido que reconvertirse, digitalizando sus negocios y la gestión de cualquier área de los mismos.

Una de las profesiones que más ha tenido que reconvertirse durante la pandemia es la de la enseñanza. A pesar de que ya existían varias aplicaciones, mencionadas más adelante en este documento, que permitían la gestión de diversas áreas de la docencia, muchas de ellas no estaban totalmente preparadas para el desempeño total de la profesión, o eran altamente mejorables. Es en este sentido donde empezaron a usarse herramientas que no estaban originalmente pensadas para un entorno académico, como Microsoft Teams, debido a que los centros docentes no tenían la tecnología adecuada, ni estaban preparados para realizar clases on-line\cite{fletcher2020digital}. A la vez, las aplicaciones que ya existían empezaron a lanzar nuevas funcionalidades para hacer frente a la demanda.

Sin embargo, muchos profesores siguen sin usar estas herramientas, o usan herramientas que no pensadas para el uso que le dan, simplemente por comodidad, porque las aplicaciones académicas no son especialmente usables, o bien por rechazo a la tecnología, a pesar de los grandes cambios tecnológicos hacia la transformación digital que se han sufrido durante la pandemia mundial. Sin ir más lejos, la mayoría de los maestros y profesores todavía usan hojas de cálculo para almacenar las calificaciones de sus alumnos. Esto, aunque sigue siendo una mejor opción que hacerlo a mano, todavía dista de ser una herramienta usable y cómoda, además de no ser adecuada para la funcionalidad que se le pretende asignar.

La mayoría de aplicaciones informáticas que existen actualmente intentan abarcar la mayoría de situaciones que se dan en un entorno de enseñanza como cuestionarios, contenidos a la carta, vídeos a modo de clases online, foros donde interactuar con los alumnos para que presenten sus dudas, publicación de notas oficiales, comunicación con las familias de los estudiantes, control de asistencia, etc. Tantas características pueden hacer que la usabilidad disminuya, al hacer falta un mayor conocimiento para poder llegar a utilizar todas las funcionalidades que ofrecen.


\section{Motivación}

Es en este contexto donde pueden aparecer nuevas herramientas que exploten las carencias de las herramientas ya existentes, que no fueron pensadas en un principio para un entorno completamente digital, y que han tenido que reinventarse para adaptarse a la transformación digital que estamos viviendo. 

Los alumnos, ya sea porque cada generación están más acostumbrados a la tecnología, ya que nacen y se desarrollan con ella, no han sufrido dificultades a la hora de adaptarse a las nuevas tecnologías y formas de estudio durante el aislamiento que sufrimos el año 2020 \cite{bogdandy2020digital}. Sin embargo, los docentes sí, tal y como se ha mencionado anteriormente, y podría deberse por varias razones.

Este trabajo plantea el desarrollo de una herramienta que permita a estos docentes, de manera fácil e intuitiva, para evitar generar rechazo, trabajar con las calificaciones de sus alumnos sin salirse del nuevo marco tecnológico. Con el fin de limitar su alcance, va dirigido exclusivamente a docentes de \gls{eso}, y consiste en la creación de una herramienta software donde puedan almacenar los datos de sus alumnos así como sus calificaciones a lo largo del curso. Para ello, se quiere crear una aplicación que, si bien permitirá a los profesionales docentes hacer el seguimiento de las calificaciones de sus alumnos, lo hará de forma simple e intuitiva, focalizando el desarrollo en que el flujo del usuario sea simple, y el tiempo requerido para la tarea sea el menor posible.


\section{Estructura del documento}
El resto del documento se estructura de la siguiente manera:

\begin{itemize}
    \item Capítulo \ref{cap:objetivo}: Objetivo. Una descripción más detallada del objetivo que tiene este proyecto.
    \item Capítulo \ref{cap:aplicacionesexistentes}: Aplicaciones existentes. Se presenta un pequeño estudio con tres de las aplicaciones existentes para la gestión de calificaciones del alumnado.
    \item Capítulo \ref{cap:metodologia}: Metodología y método de trabajo. Una explicación de la metodología de desarrollo que ha seguido el proyecto, así como el marco tecnológico de trabajo sobre el que se ha desarrollado su análisis de costes.
    \item Capítulo \ref{cap:resultados}: Resultados. Una revisión exhaustiva de la solución propuesta en este trabajo mediante imágenes y explicaciones del mismo, además de los resultados de la realización de una prueba de usabilidad, y las decisiones de diseño dignas de mencionar que se han ido tomando a lo largo del proyecto.
    \item Capítulo \ref{cap:conclusiones}: Conclusiones. En este capítulo se presenta un resumen de las conclusiones finales que se han podido obtener con este trabajo, así como propuestas para mejorar el prototipo obtenido en los resultados.
\end{itemize}
