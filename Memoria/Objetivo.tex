\chapter{Objetivo}
\label{cap:objetivo}

En este capítulo se describen los objetivos del proyecto.
\todo{desarrollar objetivos?}

\section{Objetivos}
\begin{itemize}
	\item Desarrollo de un asistente para la elección de asignatura, nivel educativo e 		ítems de calificación por evaluaciones, junto con sus pesos de valoración. 
	\item Desarrollo de un módulo de conexión entre el asistente y la herramienta de hoja de cálculo para la carga y manipulación (edición, modificación y eliminación) de los datos y las calificaciones del alumnado, permitiendo la separación por cursos y asignaturas.
	\item Desarrollo de un módulo de conexión para la creación de informes, tanto de forma individualizada por cada estudiante, como global por cada curso.
	\item Controlar la complejidad computacional del sistema desarrollado.
	\item Evaluación de la aplicación desarrollada por varios profesores de ESO.
\end{itemize}	


\section{Funcionalidades de la aplicación} 
\label{funcionalidades}

La aplicación permitirá al profesorado:
\begin{enumerate}
	\item \textbf{Iniciar sesión} mediante un usuario y una contraseña únicos para cada maestro.
	\item \textbf{Gestionar y calificar las distintas tareas y pruebas de los alumnos}. Los maestros podrán calificar los exámenes, pruebas y tareas de sus alumnos. La aplicación calculará las notas finales dependiendo de la ponderación que se le dé a las tareas. También se gestionarán las competencias que tiene cada asignatura y cada prueba.
	\item \textbf{Gestionar sus alumnos}. Cada alumno pertenece a un curso y a varias asignaturas. Todo esto se gestionará gracias a la base de datos.
	\item \textbf{Gestionar las competencias de las asignaturas}. Cada asignatura tendrá unas competencias con sus ponderaciones. Una prueba tendrá unas competencias determinadas, relacionadas con la asignatura. Esto también se gestiona desde la base de datos.
	\item \textbf{Añadir comentarios a los alumnos} sobre su asistencia, comportamiento, o cualquier punto que el maestro considere oportuno.
	\item \textbf{Visualizar informes del alumno} para ver su trayectoria escolar durante todo el año.
	\item \textbf{Visualizar informes del curso} para ver la trayectoria de los alumnos en cada trimestre.
	

\end{enumerate}

