%--- Ajustes del documento.
\pagestyle{plain}	% Páginas sólo con numeración inferior al pie

% -------------------------
%
% RESUMEN:
% OJO: Si es preciso cambiar manualmente orden Resumen <-> Abstract
%
% -------------------------
%--- Resumen en español
\selectlanguage{spanish} % Selección de idioma del resumen.
\cleardoublepage % Se incluye para modificar el contador de página antes de añadir bookmark
\phantomsection  % OJO: Necesario con hyperref
\pdfbookmark[0]{Resumen}{idx_resumen}% idx_resumen.0 % Bookmark en PDF

\begin{abstract}
% EDITAR: Resumen (máx. 1 pág.)
\begin{center}
\emph{TeachHelper}
\end{center}
Este Trabajo de Fin de Grado consiste en el desarrollo de una aplicación para escritorio que apuesta por la comodidad y conveniencia de los maestros a la hora de almacenar las calificaciones de sus alumnos.
\newline
Dicha aplicación, desarrollada en Java mediante una metodología en cascada, consistirá en un asistente que el maestro ejecutará para guardar las pruebas, notas, competencias y observaciones de sus alumnos, ordenados por asignaturas y cursos.
\newline

\end{abstract}
%---

%--- Ajuste del idioma para el resto del documento.
\ifspanish
	\selectlanguage{spanish}% Emplea idioma español
\else
	\selectlanguage{english}% Emplea idioma inglés
\fi
